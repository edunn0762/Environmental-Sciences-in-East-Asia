\chapter{Wastewater Management for a Circular Economy}
\chapterauthor{Matias Ceccarelli}

\section{Overview}

The First Law of Thermodynamics states: ''energy can be neither created nor destroyed, but can be transformed from one form to another.'' The energy constituting Earth’s resources is constant but always changing form. By understanding this law we can begin to outline sustainable waste management systems and shift from a linear extraction based/single-use economy, to a regenerative circular economy that sustains human/environmental health. While every human activity generates waste, in nature, waste does not exist. Nature operates as a circular system, turning decay into energy; nature is the model for a circular economy; it inspires biomimetic systems that function as a whole in synergy with the environment. 

The majority of the world uses a linear supply chain model, ending with disposal rather than reusal. Circularity is gaining enormous traction as the best way to sustain life on Earth. The United Nations and international governments are echoing the need to transition from a linear to a circular economy. Circularity is a way of life still practiced by first nation peoples around the world: ''do not take more than you need,'' ''replenish what is taken'' (Sharp, 1). 

Waste management is an important yet often forgotten sector underlying all systems. It is the foundation of a circular economy. It’s not within this paper’s scope to cover the intricacies of the supply chain, waste management, or circular economics; covered here is an analysis of linear and circular wastewater treatment systems that process sustainable and unsustainable waste from household, industrial, and agricultural facilities. This paper agrees, linear-waste is unsustainable and a danger to humans and the environment (Korhonen, 37; McDonoug, 80-117; CSU, 5; Gihiandelli, 1; Mang, 1). This paper postulates that if sustainable-waste is processed by bio-regenerative-wastewater-treatment-systems, defined by ingredient recovery apparatus, a circular economy will be more readily realized and humans as well as the environment will prosper. It is hoped this analysis will elucidate the interdependence of all systems, human and nonhuman, through life's fundamental element––water. This analysis is organized by multiple case studies examining distinct linear and circular wastewater treatment systems in SouthEast Asia. Here, linear wastewater treatment includes: open defecation, septic latrines, and the majority of municipal sewage waste treatment facilities; Linear waste includes pesticides and chemicals; Circular wastewater treatment includes: double-vault-compost latrines, anaerobic digesters with biogas recovery, and phytoremediation systems. Linear ends with waste; circular is wasteless. 

\subsection{Linear Economy}

The primary challenge of sustainable development comes from the energy/material stream between humans and the environment. The current and traditional linear production model: ''take, make, [use], waste,'' is unsustainable and originates from the Industrialization Revolution (1760-1840) (Korhonen, 37). The linear-supply-chain is fueled by the desire to make products as fast as possible, produce the greatest number of goods, and deliver them to the highest number of people. This process is often extractive and harmful to the environment (convenience economics). Standardized mass-production regards Earth's limited resources as endless; reuse is not a function, instead, disposal is the outcome. Pollutants generated are harmful to humans and the environment. The linear system is a major source of today’s socio-economic injustices; it is the source of climate change (McDonough, 21-24). 

The linear system begins with raw resource extraction and ends with waste disposal. Pollution is created, emissions are released, and valuable materials are lost. Since this system doesn’t restore what is extracted, the results are: 1. resource scarcity and 2. dangerous waste substances are released into land, water, and air, causing harm to humans and the environment. As a result of the linear system, quantitative geo analysis shows Earth’s usable surface area is diminishing in size and volume (Korhonen, 37). Additionally, waste emissions are emitting greenhouse gases (GHG), which are expanding deserts, causing sea level rise, changing climates, and causing the reduction of biodiversity and the extinction of species. Furthermore, unsustainable industrial and agricultural chemicals and heavy metal waste are accumulating in the environment, causing harm to humans, animals, and plants; rapid population growth is exacerbating these issues. 

\subsection{Circular Economy}

''Nature operates as a system of nutrients and metabolisms. In nature there is no waste'' (McDonoug, 92). In circular economics (CE), waste is regarded as a resource, just like in nature. Waste treatment is the metabolization process, which produces nutrients that can be repurposed. Materials can be divided into biological and technical mass (technical being industrial). Biological nutrients which contribute to the biosphere return to the Earth, while technical mass becomes nutrients to the ''technosphere'' (93). Thus, CE treats all forms of waste as food for biological and technical systems. Before being disposed of, materials are recovered for reuse, refurbishment and repair, then for remanufacturing and only later for raw material utilization (94). However, some materials used today (chemicals, plastics, etc) engage with the bio and technosphere simultaneously, often causing harm to organisms in both spheres; many argue these alien materials should be made bio-technically neutral (McDonough, 2002). 

According to CE, combustion for energy comes before landfill disposal, thus energy stored in materials can be used and not ''wasted.'' With resource recovery, a material’s value and quality is sustained for the longest possible. Energy used for resource recovery is expected to be energy efficient. The Circular Economy is intended to utilize present natural systems for preserving materials/energy in a form which nature can use in its own systems. 


\section{Wastewater Systems}

\subsection{Septic Systems: Linear}

\subsection{Vietnam’s Double-Vault Latrine: A Circular Treatment Comparison To Septic Systems }

\subsection{Municipal Utility Sewage Treatment}

\subsection{The Anaerobic Digester and Biogas: Circular Sewage Treatment}

\subsection{Phytoremediation: Wastewater Circularity}

\section{Linear Chemicals in Wastewater: Pesticides in Taiwan}

\subsection{Pesticides in Taipei Food Toxicology: (further research is needed for drinking water)}

\subsection{Circular Biopesticides: An Alternative To Hazardous Synthetic Pesticides}

\section{Further Research Needed: Linear Chemical Waste一“Forever Chemicals"}

\subsection{Household Products}

\section{Conclusion} 
